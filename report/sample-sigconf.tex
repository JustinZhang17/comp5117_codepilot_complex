%%
%% This is file `sample-sigconf.tex',
%% generated with the docstrip utility.
%%
%% The original source files were:
%%
%% samples.dtx  (with options: `all,proceedings,bibtex,sigconf')
%% 
%% IMPORTANT NOTICE:
%% 
%% For the copyright see the source file.
%% 
%% Any modified versions of this file must be renamed
%% with new filenames distinct from sample-sigconf.tex.
%% 
%% For distribution of the original source see the terms
%% for copying and modification in the file samples.dtx.
%% 
%% This generated file may be distributed as long as the
%% original source files, as listed above, are part of the
%% same distribution. (The sources need not necessarily be
%% in the same archive or directory.)
%%
%%
%% Commands for TeXCount
%TC:macro \cite [option:text,text]
%TC:macro \citep [option:text,text]
%TC:macro \citet [option:text,text]
%TC:envir table 0 1
%TC:envir table* 0 1
%TC:envir tabular [ignore] word
%TC:envir displaymath 0 word
%TC:envir math 0 word
%TC:envir comment 0 0
%%
%%
%% The first command in your LaTeX source must be the \documentclass
%% command.
%%
%% For submission and review of your manuscript please change the
%% command to \documentclass[manuscript, screen, review]{acmart}.
%%
%% When submitting camera ready or to TAPS, please change the command
%% to \documentclass[sigconf]{acmart} or whichever template is required
%% for your publication.
%%
%%
\documentclass[sigconf]{acmart}

%%
%% \BibTeX command to typeset BibTeX logo in the docs
\AtBeginDocument{%
  \providecommand\BibTeX{{%
    Bib\TeX}}}

%% Rights management information.  This information is sent to you
%% when you complete the rights form.  These commands have SAMPLE
%% values in them; it is your responsibility as an author to replace
%% the commands and values with those provided to you when you
%% complete the rights form.
\setcopyright{none}
\copyrightyear{2024}
\acmYear{2024}
\acmDOI{XXXXXXX.XXXXXXX}

%% These commands are for a PROCEEDINGS abstract or paper.
\acmConference[COMP 5117]{Mining Software Repositories}{December 2024}{Ottawa, ON}
%%
%%  Uncomment \acmBooktitle if the title of the proceedings is different
%%  from ``Proceedings of ...''!
%%
\acmBooktitle{Mining Software Repositories, December 2024, Ottawa, ON}
\acmISBN{XXX-X-XXXX-XXXX-X/24/12}


%%
%% Submission ID.
%% Use this when submitting an article to a sponsored event. You'll
%% receive a unique submission ID from the organizers
%% of the event, and this ID should be used as the parameter to this command.
%%\acmSubmissionID{123-A56-BU3}

%%
%% For managing citations, it is recommended to use bibliography
%% files in BibTeX format.
%%
%% You can then either use BibTeX with the ACM-Reference-Format style,
%% or BibLaTeX with the acmnumeric or acmauthoryear sytles, that include
%% support for advanced citation of software artefact from the
%% biblatex-software package, also separately available on CTAN.
%%
%% Look at the sample-*-biblatex.tex files for templates showcasing
%% the biblatex styles.
%%

%%
%% The majority of ACM publications use numbered citations and
%% references.  The command \citestyle{authoryear} switches to the
%% "author year" style.
%%
%% If you are preparing content for an event
%% sponsored by ACM SIGGRAPH, you must use the "author year" style of
%% citations and references.
%% Uncommenting
%% the next command will enable that style.
%%\citestyle{acmauthoryear}


%%
%% end of the preamble, start of the body of the document source.
\begin{document}

%%
%% The "title" command has an optional parameter,
%% allowing the author to define a "short title" to be used in page headers.
\title{Assessing CodePilot's Breadth}

%%
%% The "author" command and its associated commands are used to define
%% the authors and their affiliations.
%% Of note is the shared affiliation of the first two authors, and the
%% "authornote" and "authornotemark" commands
%% used to denote shared contribution to the research.
\author{Justin Zhang}
\authornote{Both authors contributed equally to this research.}
\email{JustinZhang@cmail.carleton.ca}
\affiliation{
  \institution{Carleton University}
  \city{Ottawa}
  \state{Ontario}
  \country{Canada}
}
\author{Kaya Gouin}
\authornotemark[1]
\email{KayaGouin@cmail.carleton.ca}
\affiliation{
  \institution{Carleton University}
  \city{Ottawa}
  \state{Ontario}
  \country{Canada}
}

%%
%% By default, the full list of authors will be used in the page
%% headers. Often, this list is too long, and will overlap
%% other information printed in the page headers. This command allows
%% the author to define a more concise list
%% of authors' names for this purpose.
%\renewcommand{\shortauthors}{Trovato et al.}

%%
%% The abstract is a short summary of the work to be presented in the
%% article.
\begin{abstract}
...
\end{abstract}

%%
%% The code below is generated by the tool at http://dl.acm.org/ccs.cfm.
%% Please copy and paste the code instead of the example below.
%%
\begin{CCSXML}
<ccs2012>
   <concept>
       <concept_id>10011007</concept_id>
       <concept_desc>Software and its engineering</concept_desc>
       <concept_significance>300</concept_significance>
       </concept>
 </ccs2012>
\ccsdesc[300]{Software and its engineering}
\end{CCSXML}

%%
%% Keywords. The author(s) should pick words that accurately describe
%% the work being presented. Separate the keywords with commas.
\keywords{Software engineering, prompt engineering, artificial intelligence, few-shot}
%% A "teaser" image appears between the author and affiliation
%% information and the body of the document, and typically spans the
%% page.
\iffalse
\begin{teaserfigure}
  \includegraphics[width=\textwidth]{sampleteaser}
  \caption{Seattle Mariners at Spring Training, 2010.}
  \Description{Enjoying the baseball game from the third-base
  seats. Ichiro Suzuki preparing to bat.}
  \label{fig:teaser}
\end{teaserfigure}
\fi

%%
%% This command processes the author and affiliation and title
%% information and builds the first part of the formatted document.
\maketitle

\section{Introduction}
LLMs, code coverage, CodePilot. Why we care.

\section{Assessing CodePilot's Breadth}
Introduce our own work, and the motivation behind it, by talking about the fact that CodePilot is very new and has only been tested with Python, so very limited. Here we want to explore the breadth of CodePilot by testing it on different types of programming languages. We are interested in assessing both the quality of the plan that the model outputs, the quality of the code coverage prediction that the model outputs, and the link between the two (i.e. does it look like the model uses its own plan to predict code coverage). These are our research questions.

\section{Methodology}
Five different programming languages, each belonging to a different type of language class. Onephase. Two-shot, so we give two exemplars to the model. Two tests for each language. We match the exemplar and test language. We wanted small exemplars with some level of complexity. The two exemplars for all languages are implementations of the same algorithm, one with a certain level of complexity. The two tests for all languages are implementations of the same algorithm, one with a certain level of complexity. Talk about the exact algorithms. We run each experiment once. Temperature of 0.6. GPT instruct 3.5. The template that we use (the same as in CodePilot's original paper).

\begin{figure}[h]
\centering
\frame{\includegraphics[width=\linewidth]{template_figure_final.pdf}}
\caption{Template}
\Description{template}
\end{figure}

Nam id fermentum dui. Suspendisse sagittis tortor a nulla mollis, in
pulvinar ex pretium. Sed interdum orci quis metus euismod, et sagittis
enim maximus. Vestibulum gravida massa ut felis suscipit
congue. Quisque mattis elit a risus ultrices commodo venenatis eget
dui. Etiam sagittis eleifend elementum.

\begin{figure}[h]
\centering
\frame{\includegraphics[width=\linewidth]{test_figure_final.pdf}}
\caption{Test}
\Description{test}
\end{figure}

Nam id fermentum dui. Suspendisse sagittis tortor a nulla mollis, in
pulvinar ex pretium. Sed interdum orci quis metus euismod, et sagittis
enim maximus. Vestibulum gravida massa ut felis suscipit
congue. Quisque mattis elit a risus ultrices commodo venenatis eget
dui. Etiam sagittis eleifend elementum.

\begin{figure}[h]
\centering
\frame{\includegraphics[width=\linewidth]{output_figure_final.pdf}}
\caption{Output}
\Description{output}
\end{figure}

Nam id fermentum dui. Suspendisse sagittis tortor a nulla mollis, in
pulvinar ex pretium. Sed interdum orci quis metus euismod, et sagittis
enim maximus. Vestibulum gravida massa ut felis suscipit
congue. Quisque mattis elit a risus ultrices commodo venenatis eget
dui. Etiam sagittis eleifend elementum.

\begin{figure}[h]
\centering
\frame{\includegraphics[width=\linewidth]{exemplar_figure_final.pdf}}
\caption{Exemplar}
\Description{exemplar}
\end{figure}

Nam id fermentum dui. Suspendisse sagittis tortor a nulla mollis, in
pulvinar ex pretium. Sed interdum orci quis metus euismod, et sagittis
enim maximus. Vestibulum gravida massa ut felis suscipit
congue. Quisque mattis elit a risus ultrices commodo venenatis eget
dui. Etiam sagittis eleifend elementum.

\section{Results and Discussion}
Achieved results, and our interpretation of them.

\section{Conclusion and Future Directions}
Summary of main contributions. Why this study matters, how can our findings be used. Generally speaking we see that prompt and exemplar engineering is necessary in the sense that it really has an impact on the model's output, and they must be tailored to the specific task, the specific language.

\iffalse
Modifying the template --- including but not limited to: adjusting
margins, typeface sizes, line spacing, paragraph and list definitions,
and the use of the \verb|\vspace| command to manually adjust the
vertical spacing between elements of your work --- is not allowed.

\section{Title Information}

The title of your work should use capital letters appropriately -
\url{https://capitalizemytitle.com/} has useful rules for
capitalization. Use the {\verb|title|} command to define the title of
your work. If your work has a subtitle, define it with the
{\verb|subtitle|} command.  Do not insert line breaks in your title.

If your title is lengthy, you must define a short version to be used
in the page headers, to prevent overlapping text. The \verb|title|
command has a ``short title'' parameter:
\begin{verbatim}
  \title[short title]{full title}
\end{verbatim}

\section{Sectioning Commands}

Your work should use standard \LaTeX\ sectioning commands:
\verb|section|, \verb|subsection|, \verb|subsubsection|, and
\verb|paragraph|. They should be numbered; do not remove the numbering
from the commands.

\section{Tables}

The ``\verb|acmart|'' document class includes the ``\verb|booktabs|''
package --- \url{https://ctan.org/pkg/booktabs} --- for preparing
high-quality tables.

Table captions are placed {\itshape above} the table.

Because tables cannot be split across pages, the best placement for
them is typically the top of the page nearest their initial cite.  To
ensure this proper ``floating'' placement of tables, use the
environment \textbf{table} to enclose the table's contents and the
table caption.  The contents of the table itself must go in the
\textbf{tabular} environment, to be aligned properly in rows and
columns, with the desired horizontal and vertical rules.  Again,
detailed instructions on \textbf{tabular} material are found in the
\textit{\LaTeX\ User's Guide}.

Immediately following this sentence is the point at which
Table~\ref{tab:freq} is included in the input file; compare the
placement of the table here with the table in the printed output of
this document.

\begin{table}
  \caption{Frequency of Special Characters}
  \label{tab:freq}
  \begin{tabular}{ccl}
    \toprule
    Non-English or Math&Frequency&Comments\\
    \midrule
    \O & 1 in 1,000& For Swedish names\\
    $\pi$ & 1 in 5& Common in math\\
    \$ & 4 in 5 & Used in business\\
    $\Psi^2_1$ & 1 in 40,000& Unexplained usage\\
  \bottomrule
\end{tabular}
\end{table}

To set a wider table, which takes up the whole width of the page's
live area, use the environment \textbf{table*} to enclose the table's
contents and the table caption.  As with a single-column table, this
wide table will ``float'' to a location deemed more
desirable. Immediately following this sentence is the point at which
Table~\ref{tab:commands} is included in the input file; again, it is
instructive to compare the placement of the table here with the table
in the printed output of this document.

\begin{table*}
  \caption{Some Typical Commands}
  \label{tab:commands}
  \begin{tabular}{ccl}
    \toprule
    Command &A Number & Comments\\
    \midrule
    \texttt{{\char'134}author} & 100& Author \\
    \texttt{{\char'134}table}& 300 & For tables\\
    \texttt{{\char'134}table*}& 400& For wider tables\\
    \bottomrule
  \end{tabular}
\end{table*}

Always use midrule to separate table header rows from data rows, and
use it only for this purpose. This enables assistive technologies to
recognise table headers and support their users in navigating tables
more easily.

\section{Math Equations}
You may want to display math equations in three distinct styles:
inline, numbered or non-numbered display.  Each of the three are
discussed in the next sections.

\subsection{Inline (In-text) Equations}
A formula that appears in the running text is called an inline or
in-text formula.  It is produced by the \textbf{math} environment,
which can be invoked with the usual
\texttt{{\char'134}begin\,\ldots{\char'134}end} construction or with
the short form \texttt{\$\,\ldots\$}. You can use any of the symbols
and structures, from $\alpha$ to $\omega$, available in
\LaTeX~\cite{Lamport:LaTeX}; this section will simply show a few
examples of in-text equations in context. Notice how this equation:
\begin{math}
  \lim_{n\rightarrow \infty}x=0
\end{math},
set here in in-line math style, looks slightly different when
set in display style.  (See next section).

\subsection{Display Equations}
A numbered display equation---one set off by vertical space from the
text and centered horizontally---is produced by the \textbf{equation}
environment. An unnumbered display equation is produced by the
\textbf{displaymath} environment.

Again, in either environment, you can use any of the symbols and
structures available in \LaTeX\@; this section will just give a couple
of examples of display equations in context.  First, consider the
equation, shown as an inline equation above:
\begin{equation}
  \lim_{n\rightarrow \infty}x=0
\end{equation}
Notice how it is formatted somewhat differently in
the \textbf{displaymath}
environment.  Now, we'll enter an unnumbered equation:
\begin{displaymath}
  \sum_{i=0}^{\infty} x + 1
\end{displaymath}
and follow it with another numbered equation:
\begin{equation}
  \sum_{i=0}^{\infty}x_i=\int_{0}^{\pi+2} f
\end{equation}
just to demonstrate \LaTeX's able handling of numbering.

\section{Citations and Bibliographies}

The use of \BibTeX\ for the preparation and formatting of one's
references is strongly recommended. Authors' names should be complete
--- use full first names (``Donald E. Knuth'') not initials
(``D. E. Knuth'') --- and the salient identifying features of a
reference should be included: title, year, volume, number, pages,
article DOI, etc.

The bibliography is included in your source document with these two
commands, placed just before the \verb|\end{document}| command:
\begin{verbatim}
  \bibliographystyle{ACM-Reference-Format}
  \bibliography{bibfile}
\end{verbatim}
where ``\verb|bibfile|'' is the name, without the ``\verb|.bib|''
suffix, of the \BibTeX\ file.

Citations and references are numbered by default. A small number of
ACM publications have citations and references formatted in the
``author year'' style; for these exceptions, please include this
command in the {\bfseries preamble} (before the command
``\verb|\begin{document}|'') of your \LaTeX\ source:
\begin{verbatim}
  \citestyle{acmauthoryear}
\end{verbatim}


  Some examples.  A paginated journal article \cite{Abril07}, an
  enumerated journal article \cite{Cohen07}, a reference to an entire
  issue \cite{JCohen96}, a monograph (whole book) \cite{Kosiur01}, a
  monograph/whole book in a series (see 2a in spec. document)
  \cite{Harel79}, a divisible-book such as an anthology or compilation
  \cite{Editor00} followed by the same example, however we only output
  the series if the volume number is given \cite{Editor00a} (so
  Editor00a's series should NOT be present since it has no vol. no.),
  a chapter in a divisible book \cite{Spector90}, a chapter in a
  divisible book in a series \cite{Douglass98}, a multi-volume work as
  book \cite{Knuth97}, a couple of articles in a proceedings (of a
  conference, symposium, workshop for example) (paginated proceedings
  article) \cite{Andler79, Hagerup1993}, a proceedings article with
  all possible elements \cite{Smith10}, an example of an enumerated
  proceedings article \cite{VanGundy07}, an informally published work
  \cite{Harel78}, a couple of preprints \cite{Bornmann2019,
    AnzarootPBM14}, a doctoral dissertation \cite{Clarkson85}, a
  master's thesis: \cite{anisi03}, an online document / world wide web
  resource \cite{Thornburg01, Ablamowicz07, Poker06}, a video game
  (Case 1) \cite{Obama08} and (Case 2) \cite{Novak03} and \cite{Lee05}
  and (Case 3) a patent \cite{JoeScientist001}, work accepted for
  publication \cite{rous08}, 'YYYYb'-test for prolific author
  \cite{SaeediMEJ10} and \cite{SaeediJETC10}. Other cites might
  contain 'duplicate' DOI and URLs (some SIAM articles)
  \cite{Kirschmer:2010:AEI:1958016.1958018}. Boris / Barbara Beeton:
  multi-volume works as books \cite{MR781536} and \cite{MR781537}. A
  couple of citations with DOIs:
  \cite{2004:ITE:1009386.1010128,Kirschmer:2010:AEI:1958016.1958018}. Online
  citations: \cite{TUGInstmem, Thornburg01, CTANacmart}.
  Artifacts: \cite{R} and \cite{UMassCitations}.

%%
%% The acknowledgments section is defined using the "acks" environment
%% (and NOT an unnumbered section). This ensures the proper
%% identification of the section in the article metadata, and the
%% consistent spelling of the heading.

\begin{acks}
To Robert, for the bagels and explaining CMYK and color spaces.
\end{acks}

%%
%% The next two lines define the bibliography style to be used, and
%% the bibliography file.
\bibliographystyle{ACM-Reference-Format}
\bibliography{sample-base}


%%
%% If your work has an appendix, this is the place to put it.
\appendix

\section{Appendix}

\subsection{Part One}
...

\subsection{Part Two}
...

\newpage
\section{Notes}
\begin{verbatim}
Input:

For the given code snippet, predict the code coverage.
The code coverage indicates whether a statement has been
executed or not.

> if the line is executed
! if the line is not executed

Example output:
> line1
! line2
> line3
...
> linen

You need to develop a plan for step by step execution of
the code snippet.

Do not answer unless instructed to do so

DISCLAIMER: Lines that are not executed are to be denoted
with a SINGLE '!' whereas lines that are executed are to be
denoted with a single '>'

Below are a couple examples of the process you need to follow
to predict the code coverage of a given code snippet and its
plan.

[example 1]

[example 2]

In a similar fashion, develop a PLAN of step by step execution
of the below code snippet and predict the CODE COVERAGE.

[code snippet]

Output:

[output plan]
[output code coverage]

\end{verbatim}

\newpage
%exemplar
\begin{verbatim}
Given CODE SNIPPET:
isValidParentheses :: String -> Bool
isValidParentheses str = check str 0
  where
    check [] 0 = True
    check [] _ = False
    check (')':xs) n = n > 0 && check xs (n - 1)
    check ('(':xs) n = check xs (n + 1)
    check (_:xs) n = check xs n
main = do
  let result = isValidParentheses "(()"
  print result

Given PLAN: 
STEP 1: Define a function isValidParentheses which
takes a string as input and produces a boolean as output.
STEP 2: In the main function, let the "result" variable
hold the output of isValidParentheses given the string
"(()".
STEP 3: In the isValidParentheses function, initiate the
conditional checks with parameters str and 0.
STEP 4: Define five checks. Check 1 verifies whether the
parameters are an empty list and the value 0. Check 2 verifies
whether the parameters are an empty list and any value other
than 0. Check 3 verifies whether the first character of the
first parameter is a right parenthesis. Check 4 verifies
whether the first character of the first parameter is a left
parenthesis. Check 5 verifies whether the first character of
the first parameter is any character other than a right or a
left parenthesis. These checks are executed in the order in
which they appear. We stop executing the next check as soon as
we find one which satisfies the correct condition. In the case
of the "(()" input string, given the ordering of the checks,
all checks are executed.
STEP 5: In the main function, print the resulting array to the
screen.

So the CODE COVERAGE for the given code snippet will be: 
> isValidParentheses :: String -> Bool
> isValidParentheses str = check str 0
>   where
>     check [] 0 = True
>     check [] _ = False
>     check (')':xs) n = n > 0 && check xs (n - 1)
>     check ('(':xs) n = check xs (n + 1)
>     check (_:xs) n = check xs n
> main = do
>   let result = isValidParentheses "(()"
>   print result
\end{verbatim}

%test
\begin{verbatim}
maxProfit :: [Int] -> Int
maxProfit prices = maxProfitHelper prices (maxBound :: Int) 0
  where
    maxProfitHelper [] _ maxProfit = maxProfit
    maxProfitHelper (p:ps) minPrice maxProfit =
      let newMinPrice = min minPrice p
          profit = p - newMinPrice
          newMaxProfit = max maxProfit profit
      in maxProfitHelper ps newMinPrice newMaxProfit
main = do
    let prices = [7, 1, 5, 3, 6, 4]
    let result = maxProfit prices
    putStrLn $ "Maximum profit: " ++ show result
\end{verbatim}

%output
\begin{verbatim}
Given PLAN:
STEP 1: Define a function maxProfit which takes an array of
integers as input and produces an integer as output.
STEP 2: In the main function, let the "prices" variable hold
the input array of integers [7, 1, 5, 3, 6, 4].
STEP 3: In the main function, let the "result" variable hold
the output of maxProfit given the input array of integers.
STEP 4: In the maxProfit function, initiate the conditional
checks with parameters prices, (maxBound :: Int), and 0.
STEP 5: Define two checks. Check 1 verifies whether the first
parameter is an empty list and the second parameter is the
maximum bound of the integer type. Check 2 verifies whether
the first parameter is a non-empty list. These checks are
executed in the order in which they appear. In the case of the
[7, 1, 5, 3, 6, 4] input array, given the ordering of the checks,
only check 2 is executed.
STEP 6: In the maxProfit function, define a helper function
maxProfitHelper which takes three parameters: an array of
integers, an integer representing the minimum price, and an
integer representing the maximum profit. This helper function
recursively calculates the maximum profit by updating the
minimum price and maximum profit values based on the current
element of the input array.
STEP 7: In the main function, print the resulting maximum profit
to the screen.

So the CODE COVERAGE for the given code snippet will be: 
> maxProfit :: [Int] -> Int
> maxProfit prices = maxProfitHelper prices (maxBound :: Int) 0
>   where
>     maxProfitHelper [] _ maxProfit = maxProfit
>     maxProfitHelper (p:ps) minPrice maxProfit =
>       let newMinPrice = min minPrice p
>           profit = p - newMinPrice
>           newMaxProfit = max maxProfit profit
>       in maxProfitHelper ps newMinPrice newMaxProfit
> main = do
>     let prices = [7, 1, 5, 3, 6, 4]
>     let result = maxProfit prices
>     putStrLn $ "Maximum profit: " ++ show result
\end{verbatim}
\fi

\end{document}
\endinput
%%
%% End of file `sample-sigconf.tex'.
